\documentclass[12pt]{article}
\usepackage{pmmeta}
\pmcanonicalname{ConnectedSum}
\pmcreated{2013-03-22 13:17:33}
\pmmodified{2013-03-22 13:17:33}
\pmowner{Mathprof}{13753}
\pmmodifier{Mathprof}{13753}
\pmtitle{connected sum}
\pmrecord{11}{33780}
\pmprivacy{1}
\pmauthor{Mathprof}{13753}
\pmtype{Definition}
\pmcomment{trigger rebuild}
\pmclassification{msc}{57M25}
\pmsynonym{knot sum}{ConnectedSum}
\pmrelated{KnotTheory}

\endmetadata

% this is the default PlanetMath preamble.  as your knowledge
% of TeX increases, you will probably want to edit this, but
% it should be fine as is for beginners.

% almost certainly you want these
\usepackage{amssymb}
\usepackage{amsmath}
\usepackage{amsfonts}

% used for TeXing text within eps files
%\usepackage{psfrag}
% need this for including graphics (\includegraphics)
\usepackage{graphicx}
% for neatly defining theorems and propositions
\usepackage{amsthm}
% making logically defined graphics
%%%\usepackage{xypic}
\usepackage[all,knot,poly]{xy}

% there are many more packages, add them here as you need them

% define commands here
\newcommand{\trefoil}{\xygraph{
 !{0;/r2.0pc/:}
 !P3"a"{~>{}}
 !P9"b"{~:{(1.3288,0):}~>{}}
 !P3"c"{~:{(2.5,0):}~>{}}
 !{\vunder~{"b2"}{"b1"}{"a1"}{"a3"}}
 !{\vcap~{"c1"}{"c1"}{"b4"}{"b2"}}
 !{\vunder~{"b5"}{"b4"}{"a2"}{"a1"}}
 !{\vcap~{"c2"}{"c2"}{"b7"}{"b5"}}
 !{\vunder~{"b8"}{"b7"}{"a3"}{"a2"}}
 !{\vcap~{"c3"}{"c3"}{"b1"}{"b8"}}
}}

\newcommand{\quatrefoil}{\xygraph{
 !{0;/r2.0pc/:}
 !P3"a"{~>{}}
 !P9"b"{~:{(1.3288,0):}~>{}}
 !P3"c"{~:{(2.5,0):}~>{}} [rrrr]
 !P3"d"{~>{}}
 !P9"e"{~:{(1.3288,0):}~>{}}
 !P3"f"{~:{(2.5,0):}~>{}} [rr]
 !{\vunder~{"b2"}{"b1"}{"a1"}{"a3"}}
 !{\vcap~{"c1"}{"c1"}{"b4"}{"b2"}}
 !{\vunder~{"b5"}{"b4"}{"a2"}{"a1"}}
 !{\vcap~{"c2"}{"c2"}{"b7"}{"b5"}}
 !{\vunder~{"b8"}{"b7"}{"a3"}{"a2"}}
 !{\huntwist~{"b8"}{"e7"}{"b1"}{"e5"}}
 !{\vover~{"e7"}{"e8"}{"d2"}{"d3"}}
 !{\vcap~{"f3"}{"f3"}{"e8"}{"e1"}}
 !{\vover~{"e1"}{"e2"}{"d3"}{"d1"}}
 !{\vcap~{"f1"}{"f1"}{"e2"}{"e4"}}
 !{\vover~{"e4"}{"e5"}{"d1"}{"d2"}}
}}

\newtheorem*{example*}{Example}
\begin{document}
The \emph{connected sum of knots} $K$ and $J$ is a knot, denoted by $K\#J$,
constructed by removing a short segment from each of $K$ and $J$ and joining each free end of $K$ to a different free end of $J$ to form a new knot.  The connected sum of two knots always exists but is not necessarily unique.

The \emph{connected sum of oriented knots} $K$ and $J$ is a connected sum of knots which has a consistent orientation inherited from that of $K$ and $J$.  This sum always exists and is unique.

\begin{example*}
Suppose $K$ and $J$ are both the trefoil knot.
\begin{figure}[here]
  \begin{center}
  \includegraphics{TrefoilKnotPF2007}
  \leavevmode
  \end{center}
  \caption{The trefoil knot}
\end{figure}
By one choice of segment deletion and reattachment, $K\#J$ is the quatrefoil knot.
%\quatrefoil source in quatrefoil.tex (also defined in header of this file)
\begin{figure}[here]
  \begin{center}
  \includegraphics{QuatrefoilKnotPF2007}
  \leavevmode
  \end{center}
  \caption{$K\#J$ is the quatrefoil knot}
\end{figure}
\end{example*}
%%%%%
%%%%%
\end{document}
