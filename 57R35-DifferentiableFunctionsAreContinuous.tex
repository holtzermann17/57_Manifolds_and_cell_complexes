\documentclass[12pt]{article}
\usepackage{pmmeta}
\pmcanonicalname{DifferentiableFunctionsAreContinuous}
\pmcreated{2013-03-22 14:35:27}
\pmmodified{2013-03-22 14:35:27}
\pmowner{matte}{1858}
\pmmodifier{matte}{1858}
\pmtitle{differentiable functions are continuous}
\pmrecord{8}{36154}
\pmprivacy{1}
\pmauthor{matte}{1858}
\pmtype{Theorem}
\pmcomment{trigger rebuild}
\pmclassification{msc}{57R35}
\pmclassification{msc}{26A24}
\pmrelated{DifferentiableFunctionsAreContinuous2}
\pmrelated{LimitsOfNaturalLogarithm}

\endmetadata

% this is the default PlanetMath preamble.  as your knowledge
% of TeX increases, you will probably want to edit this, but
% it should be fine as is for beginners.

% almost certainly you want these
\usepackage{amssymb}
\usepackage{amsmath}
\usepackage{amsfonts}
\usepackage{amsthm}

% used for TeXing text within eps files
%\usepackage{psfrag}
% need this for including graphics (\includegraphics)
%\usepackage{graphicx}
% for neatly defining theorems and propositions
%
% making logically defined graphics
%%%\usepackage{xypic}

% there are many more packages, add them here as you need them

% define commands here

\newcommand{\sR}[0]{\mathbb{R}}
\newcommand{\sC}[0]{\mathbb{C}}
\newcommand{\sN}[0]{\mathbb{N}}
\newcommand{\sZ}[0]{\mathbb{Z}}

 \usepackage{bbm}
 \newcommand{\Z}{\mathbbmss{Z}}
 \newcommand{\C}{\mathbbmss{C}}
 \newcommand{\R}{\mathbbmss{R}}
 \newcommand{\Q}{\mathbbmss{Q}}



\newcommand*{\norm}[1]{\lVert #1 \rVert}
\newcommand*{\abs}[1]{| #1 |}



\newtheorem{thm}{Theorem}
\newtheorem{defn}{Definition}
\newtheorem{prop}{Proposition}
\newtheorem{lemma}{Lemma}
\newtheorem{cor}{Corollary}
\begin{document}
\begin{prop}
Suppose $I$ is an open interval on $\mathbb{R}$,
and $f\colon I\to \sC$ is differentiable at $x\in I$. Then
$f$ is continuous at $x$. Further, if $f$ is differentiable on $I$,
then $f$ is continuous on $I$.
\end{prop}
                                                                                
\begin{proof}
Suppose $x\in I$. Let us show that
$f(y)\to f(x)$, when $y\to x$. First, if $y\in I$ is distinct to $x$,
then
$$
  f(x)-f(y) = \frac{f(x)-f(y)}{x-y} (x-y).
$$
Thus, if $f'(x)$ is the derivative of $f$ at $x$, we have
\begin{eqnarray*}
\lim_{y\to x} f(x)-f(y) &=& \lim_{y\to x} \frac{f(x)-f(y)}{x-y} (x-y) \\
   &=& \lim_{y\to x} \frac{f(x)-f(y)}{x-y}\  \lim_{y\to x}  (x-y) \\
   &=& f'(x)\  0 \\
   &=& 0,
\end{eqnarray*}
where the second equality is justified since both limits on the second line
exist. The second claim follows since $f$ is continuous on $I$ if and only
if $f$ is continuous at $x$ for all $x\in I$.  
\end{proof}
%%%%%
%%%%%
\end{document}
