\documentclass[12pt]{article}
\usepackage{pmmeta}
\pmcanonicalname{Isotopy}
\pmcreated{2013-03-22 14:52:49}
\pmmodified{2013-03-22 14:52:49}
\pmowner{rspuzio}{6075}
\pmmodifier{rspuzio}{6075}
\pmtitle{isotopy}
\pmrecord{5}{36558}
\pmprivacy{1}
\pmauthor{rspuzio}{6075}
\pmtype{Definition}
\pmcomment{trigger rebuild}
\pmclassification{msc}{57R52}
\pmrelated{ExampleOfMappingClassGroup}
\pmrelated{Homeotopy}

% this is the default PlanetMath preamble.  as your knowledge
% of TeX increases, you will probably want to edit this, but
% it should be fine as is for beginners.

% almost certainly you want these
\usepackage{amssymb,amscd}
\usepackage{amsmath}
\usepackage{amsfonts}

% used for TeXing text within eps files
%\usepackage{psfrag}
% need this for including graphics (\includegraphics)
%\usepackage{graphicx}
% for neatly defining theorems and propositions
%\usepackage{amsthm}
% making logically defined graphics
%%%\usepackage{xypic}

% there are many more packages, add them here as you need them

% define commands here
\begin{document}
Let $M$ and $N$ be manifolds and $I=[0,1]$ the closed unit interval.  A smooth map $h\colon M\times I\to N$ is called an \emph{isotopy} if the restriction map $h_t:=h(-,t):M\to N$ is an embedding for all $t\in I$.

In particular, a diffeotopy is an isotopy.

\textbf{Remark}.  Given an isotopy $h\colon M\times I\to N$, there exists a diffeotopy $g\colon N\times I\to N$ such that $h_t=g_t\circ h_0$.
%%%%%
%%%%%
\end{document}
