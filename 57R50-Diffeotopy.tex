\documentclass[12pt]{article}
\usepackage{pmmeta}
\pmcanonicalname{Diffeotopy}
\pmcreated{2013-03-22 14:52:43}
\pmmodified{2013-03-22 14:52:43}
\pmowner{rspuzio}{6075}
\pmmodifier{rspuzio}{6075}
\pmtitle{diffeotopy}
\pmrecord{9}{36556}
\pmprivacy{1}
\pmauthor{rspuzio}{6075}
\pmtype{Definition}
\pmcomment{trigger rebuild}
\pmclassification{msc}{57R50}
\pmdefines{isotopic}
\pmdefines{diffeotopic}

\endmetadata

% this is the default PlanetMath preamble.  as your knowledge
% of TeX increases, you will probably want to edit this, but
% it should be fine as is for beginners.

% almost certainly you want these
\usepackage{amssymb,amscd}
\usepackage{amsmath}
\usepackage{amsfonts}

% used for TeXing text within eps files
%\usepackage{psfrag}
% need this for including graphics (\includegraphics)
%\usepackage{graphicx}
% for neatly defining theorems and propositions
%\usepackage{amsthm}
% making logically defined graphics
%%%\usepackage{xypic}

% there are many more packages, add them here as you need them

% define commands here
\begin{document}
Let $M$ be a manifold and $I=[0,1]$ the closed unit interval.  A smooth map $h\colon M\times I\rightarrow M$ is called a \emph{diffeotopy} (on $M$) if for every $t\in I$: $$h_t:=h(-,t)\colon M\rightarrow M$$ is a diffeomorphism.

Two diffeomorphisms $f,g\colon M\to M$ are said to be \emph{diffeotopic} if there is a diffeotopy $h\colon M\times I\to M$ such that 
\begin{enumerate}
\item $h_0=f$, and
\item $h_1=g$.
\end{enumerate}

\textbf{Remark}.  Diffeotopy is an equivalence relation among diffeomorphisms.  In particular, those diffeomorphisms that are diffeotopic to the identity map form a group.

Two points $a,b\in M$ are said to be \emph{isotopic} if there is a diffeotopy $h$ on $M$ such that 
\begin{enumerate}
\item $h_0=id_M$, the identity map on $M$, and
\item $h_1(a)=b$.
\end{enumerate}

\textbf{Remark}.  If $M$ is a connected manifold, then every pair of points on $M$ are isotopic.

Pairs of isotopic points in a manifold can be generazlied to pairs of isotopic sets.  Two arbitrary sets $A,B\subseteq M$ are said to be \emph{isotopic} if there is a diffeotopy $h$ on $M$ such that 
\begin{enumerate}
\item $h_0=id_M$, and
\item $h_1(A)=B$.
\end{enumerate}

\textbf{Remark}.  One special example of isotopic sets is the isotopy of curves.  In $\mathbb{R}^3$, curves that are isotopic to the unit circle are the trivial knots.
%%%%%
%%%%%
\end{document}
