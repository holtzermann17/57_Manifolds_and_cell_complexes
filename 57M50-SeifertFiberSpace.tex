\documentclass[12pt]{article}
\usepackage{pmmeta}
\pmcanonicalname{SeifertFiberSpace}
\pmcreated{2013-03-22 15:50:53}
\pmmodified{2013-03-22 15:50:53}
\pmowner{juanman}{12619}
\pmmodifier{juanman}{12619}
\pmtitle{Seifert fiber space}
\pmrecord{10}{37831}
\pmprivacy{1}
\pmauthor{juanman}{12619}
\pmtype{Definition}
\pmcomment{trigger rebuild}
\pmclassification{msc}{57M50}
\pmsynonym{circle bundle}{SeifertFiberSpace}
\pmrelated{3Manifolds}

% this is the default PlanetMath preamble.  as your knowledge
% of TeX increases, you will probably want to edit this, but
% it should be fine as is for beginners.

% almost certainly you want these
\usepackage{amssymb}
\usepackage{amsmath}
\usepackage{amsfonts}

% used for TeXing text within eps files
%\usepackage{psfrag}
% need this for including graphics (\includegraphics)
%\usepackage{graphicx}
% for neatly defining theorems and propositions
%\usepackage{amsthm}
% making logically defined graphics
%%%\usepackage{xypic}

% there are many more packages, add them here as you need them

% define commands here
\begin{document}
In the field of three dimensional manifolds there is a kind which can be considered as spaces which are fibered by
\PMlinkname{circles}{Circle}, that is, for each point in the space there is a unique circle (a homeomorph of $S^1$) which contains it.

A Seifert fiber space can be defined as a circle bundle over an orbifold,
however these kind of 3-manifolds appeared much earlier than the concept of fiber-bundle.

To construct them one must begin learning how to fiber a solid torus. 
Naturally a solid torus $D^2\times S^1$ is foliated by circles $p\times S^1$ where $p\in D^2$.

To get different fibered solid tori one must begin with a solid torus $D^2\times S^1$, cut it along a meridian disk and twist the resulting solid cylinder by a rational angle $2\pi(a/b)$ (being $a,b$ coprime) and reglue to obtain a new fibering by circles of the old torus, where the fibers all are still longitudinal circles but differs by the previous one in the fact that now only the center's disk is a trivial circle but any other point in the disk is in a circle which is $b$ times longer that the central one.

\bigskip

It is possible construct an associated 2-manifold identifying each fiber to a point to get the so called {\bf orbit surface}.


\bigskip


{\bf References}

H. Seifert,{\it Topologie drei-dimensionaler gefaserter R\"aume}, Acta. Math. 60 (1933), 147 - 238.

M.Brin, {\it  Seifert Fibered Spaces, Notes for a course given in the Spring of 1993}, on line at
ftp://ftp.math.binghamton.edu/pub/matt/seifert.pdf
%%%%%
%%%%%
\end{document}
