\documentclass[12pt]{article}
\usepackage{pmmeta}
\pmcanonicalname{TopicsInAlgebraicTopology}
\pmcreated{2013-03-22 18:23:42}
\pmmodified{2013-03-22 18:23:42}
\pmowner{bci1}{20947}
\pmmodifier{bci1}{20947}
\pmtitle{topics in algebraic topology}
\pmrecord{82}{41041}
\pmprivacy{1}
\pmauthor{bci1}{20947}
\pmtype{Topic}
\pmcomment{trigger rebuild}
\pmclassification{msc}{57R19}
\pmclassification{msc}{57N65}
\pmclassification{msc}{11F23}
\pmclassification{msc}{11E72}
\pmclassification{msc}{18-00}
\pmclassification{msc}{55N30}
\pmclassification{msc}{55N15}
\pmclassification{msc}{55N99}
\pmclassification{msc}{55N40}
\pmclassification{msc}{55N20}
\pmclassification{msc}{55-01}
\pmsynonym{category theory}{TopicsInAlgebraicTopology}
\pmsynonym{algebraic geometry}{TopicsInAlgebraicTopology}
\pmsynonym{topology and groupoids}{TopicsInAlgebraicTopology}
%\pmkeywords{homology and cohomology theory}
%\pmkeywords{fundamental functor}
%\pmkeywords{fundamental groupoid functor}
%\pmkeywords{groupoid category}
%\pmkeywords{algebroid category}
%\pmkeywords{crossed complexes}
%\pmkeywords{complex modules}
%\pmkeywords{homology groups and groupoids. homotopy theory}
%\pmkeywords{groupoids}
%\pmkeywords{categorical algebra}
%\pmkeywords{topological categ}
\pmrelated{Category}
\pmrelated{Functor}
\pmrelated{GrothendieckGroup}
\pmrelated{NoncommutativeGeometry}
\pmrelated{TopologicalGroup2}
\pmrelated{Naturaltransformation}
\pmrelated{AlgebraicTopology}
\pmrelated{AlgebraicGeometry}
\pmrelated{CategoricalOntologyABibliographyOfCategoryTheory}
\pmrelated{IndexOfCategories}
\pmrelated{OverviewOfTheContentOfPlanetMath}
\pmrelated{CategoryOf}

\endmetadata

% this is the default PlanetMath preamble.  as your 
% almost certainly you want these
\usepackage{amssymb}
\usepackage{amsmath}
\usepackage{amsfonts}

% used for TeXing text within eps files
%\usepackage{psfrag}
% need this for including graphics (\includegraphics)
%\usepackage{graphicx}
% for neatly defining theorems and propositions
%\usepackage{amsthm}
% making logically defined graphics
%%%\usepackage{xypic}

% there are many more packages, add them here as you need them

% define commands here
\usepackage{amsmath, amssymb, amsfonts, amsthm, amscd, latexsym}
%%\usepackage{xypic}
\usepackage[mathscr]{eucal}

\setlength{\textwidth}{6.5in}
%\setlength{\textwidth}{16cm}
\setlength{\textheight}{9.0in}
%\setlength{\textheight}{24cm}

\hoffset=-.75in     %%ps format
%\hoffset=-1.0in     %%hp format
\voffset=-.4in

\theoremstyle{plain}
\newtheorem{lemma}{Lemma}[section]
\newtheorem{proposition}{Proposition}[section]
\newtheorem{theorem}{Theorem}[section]
\newtheorem{corollary}{Corollary}[section]

\theoremstyle{definition}
\newtheorem{definition}{Definition}[section]
\newtheorem{example}{Example}[section]
%\theoremstyle{remark}
\newtheorem{remark}{Remark}[section]
\newtheorem*{notation}{Notation}
\newtheorem*{claim}{Claim}

\renewcommand{\thefootnote}{\ensuremath{\fnsymbol{footnote%%@
}}}
\numberwithin{equation}{section}

\newcommand{\Ad}{{\rm Ad}}
\newcommand{\Aut}{{\rm Aut}}
\newcommand{\Cl}{{\rm Cl}}
\newcommand{\Co}{{\rm Co}}
\newcommand{\DES}{{\rm DES}}
\newcommand{\Diff}{{\rm Diff}}
\newcommand{\Dom}{{\rm Dom}}
\newcommand{\Hol}{{\rm Hol}}
\newcommand{\Mon}{{\rm Mon}}
\newcommand{\Hom}{{\rm Hom}}
\newcommand{\Ker}{{\rm Ker}}
\newcommand{\Ind}{{\rm Ind}}
\newcommand{\IM}{{\rm Im}}
\newcommand{\Is}{{\rm Is}}
\newcommand{\ID}{{\rm id}}
\newcommand{\GL}{{\rm GL}}
\newcommand{\Iso}{{\rm Iso}}
\newcommand{\Sem}{{\rm Sem}}
\newcommand{\St}{{\rm St}}
\newcommand{\Sym}{{\rm Sym}}
\newcommand{\SU}{{\rm SU}}
\newcommand{\Tor}{{\rm Tor}}
\newcommand{\U}{{\rm U}}

\newcommand{\A}{\mathcal A}
\newcommand{\Ce}{\mathcal C}
\newcommand{\D}{\mathcal D}
\newcommand{\E}{\mathcal E}
\newcommand{\F}{\mathcal F}
\newcommand{\G}{\mathcal G}
\newcommand{\Q}{\mathcal Q}
\newcommand{\R}{\mathcal R}
\newcommand{\cS}{\mathcal S}
\newcommand{\cU}{\mathcal U}
\newcommand{\W}{\mathcal W}

\newcommand{\bA}{\mathbb{A}}
\newcommand{\bB}{\mathbb{B}}
\newcommand{\bC}{\mathbb{C}}
\newcommand{\bD}{\mathbb{D}}
\newcommand{\bE}{\mathbb{E}}
\newcommand{\bF}{\mathbb{F}}
\newcommand{\bG}{\mathbb{G}}
\newcommand{\bK}{\mathbb{K}}
\newcommand{\bM}{\mathbb{M}}
\newcommand{\bN}{\mathbb{N}}
\newcommand{\bO}{\mathbb{O}}
\newcommand{\bP}{\mathbb{P}}
\newcommand{\bR}{\mathbb{R}}
\newcommand{\bV}{\mathbb{V}}
\newcommand{\bZ}{\mathbb{Z}}

\newcommand{\bfE}{\mathbf{E}}
\newcommand{\bfX}{\mathbf{X}}
\newcommand{\bfY}{\mathbf{Y}}
\newcommand{\bfZ}{\mathbf{Z}}

\renewcommand{\O}{\Omega}
\renewcommand{\o}{\omega}
\newcommand{\vp}{\varphi}
\newcommand{\vep}{\varepsilon}

\newcommand{\diag}{{\rm diag}}
\newcommand{\grp}{{\mathbb G}}
\newcommand{\dgrp}{{\mathbb D}}
\newcommand{\desp}{{\mathbb D^{\rm{es}}}}
\newcommand{\Geod}{{\rm Geod}}
\newcommand{\geod}{{\rm geod}}
\newcommand{\hgr}{{\mathbb H}}
\newcommand{\mgr}{{\mathbb M}}
\newcommand{\ob}{{\rm Ob}}
\newcommand{\obg}{{\rm Ob(\mathbb G)}}
\newcommand{\obgp}{{\rm Ob(\mathbb G')}}
\newcommand{\obh}{{\rm Ob(\mathbb H)}}
\newcommand{\Osmooth}{{\Omega^{\infty}(X,*)}}
\newcommand{\ghomotop}{{\rho_2^{\square}}}
\newcommand{\gcalp}{{\mathbb G(\mathcal P)}}

\newcommand{\rf}{{R_{\mathcal F}}}
\newcommand{\glob}{{\rm glob}}
\newcommand{\loc}{{\rm loc}}
\newcommand{\TOP}{{\rm TOP}}

\newcommand{\wti}{\widetilde}
\newcommand{\what}{\widehat}

\renewcommand{\a}{\alpha}
\newcommand{\be}{\beta}
\newcommand{\ga}{\gamma}
\newcommand{\Ga}{\Gamma}
\newcommand{\de}{\delta}
\newcommand{\del}{\partial}
\newcommand{\ka}{\kappa}
\newcommand{\si}{\sigma}
\newcommand{\ta}{\tau}
\newcommand{\med}{\medbreak}
\newcommand{\medn}{\medbreak \noindent}
\newcommand{\bign}{\bigbreak \noindent}
\newcommand{\lra}{{\longrightarrow}}
\newcommand{\ra}{{\rightarrow}}
\newcommand{\rat}{{\rightarrowtail}}
\newcommand{\oset}[1]{\overset {#1}{\ra}}
\newcommand{\osetl}[1]{\overset {#1}{\lra}}
\newcommand{\hr}{{\hookrightarrow}}

\begin{document}
\section{Algebraic topology topics}


\subsection{Introduction}
\emph{Algebraic topology} (AT) utilizes algebraic approaches to solve topological problems,
such as the classification of surfaces, proving duality theorems for manifolds and 
approximation theorems for topological spaces. A central problem in algebraic topology 
is to find algebraic invariants of topological spaces, which is usually carried out by means
of homotopy, homology and cohomology groups. There are close connections between algebraic topology, 
\PMlinkname{Algebraic Geometry (AG)}{AlgebraicGeometry}, and Non-commutative Geometry/NAAT. On the other hand, there are also close ties between algebraic geometry and number 
theory. 


\subsection{Outline}
\begin{enumerate}

\item Homotopy theory and fundamental groups
\item Topology and groupoids; \PMlinkname{van Kampen theorem}{VanKampensTheorem} 
\item Homology and cohomology theories
\item Duality
\item Category theory applications in algebraic topology
\item Index of categories, functors and natural transformations
\item \PMlinkexternal{Grothendieck's Descent theory}{http://www.uclouvain.be/17501.html}
\item `Anabelian geometry'
\item Categorical Galois theory
\item Higher dimensional algebra (HDA)
\item Quantum algebraic topology (QAT)
\item Quantum Geometry
\item Non-Abelian algebraic topology (NAAT)
\end{enumerate}

\subsection{Homotopy theory and fundamental groups}
\begin{enumerate}
\item Homotopy
\item Fundamental group of a space
\item Fundamental theorems
\item van Kampen theorem
\item Whitehead groups, torsion and towers
\item Postnikov towers
\end{enumerate}


\subsection{Topology and Groupoids}
\begin{enumerate}
\item Topology definition, axioms and basic concepts
\item Fundamental groupoid
\item Topological groupoid
\item van Kampen theorem for groupoids
\item Groupoid pushout theorem 
\item Double groupoids and crossed modules
\item new4

\end{enumerate}


\subsection{Homology theory}
\begin{enumerate}

\item Homology group
\item Homology sequence
\item Homology complex
\item new4

\end{enumerate}


\subsection{Cohomology theory}
\begin{enumerate}

\item Cohomology group
\item Cohomology sequence
\item DeRham cohomology
\item new4

\end{enumerate}



\subsection{Duality in algebraic topology and category theory}
\begin{enumerate}

\item Tanaka-Krein duality
\item Grothendieck duality
\item Categorical duality
\item Tangled duality
\item DA5
\item DA6
\item DA7

\end{enumerate}

\subsection{Category theory applications}
\begin{enumerate}
\item Abelian categories
\item Topological category
\item Fundamental groupoid functor
\item Categorical Galois theory
\item Non-Abelian algebraic topology
\item Group category
\item Groupoid category
\item $\mathcal{T}op$ category
\item Topos and topoi axioms
\item Generalized toposes
\item Categorical logic and algebraic topology
\item Meta-theorems
\item Duality between spaces and algebras

\end{enumerate}


\subsection{Index of categories}
The following is a listing of categories relevant to algebraic topology:

\begin{enumerate}
\item \PMlinkexternal{Algebraic categories}{http://www.uclouvain.be/17501.html}
\item Topological category
\item Category of sets, Set
\item Category of topological spaces
\item Category of Riemannian manifolds
\item Category of CW-complexes
\item Category of Hausdorff spaces
\item Category of Borel spaces
\item Category of CR-complexes
\item Category of graphs
\item Category of spin networks
\item Category of groups
\item Galois category
\item Category of fundamental groups
\item Category of Polish groups
\item Groupoid category
\item Category of groupoids (or groupoid category)
\item Category of Borel groupoids
\item Category of fundamental groupoids
\item Category of functors (or functor category)
\item Double groupoid category
\item Double category
\item Category of Hilbert spaces
\item Category of quantum automata
\item R-category
\item Category of algebroids
\item Category of double algebroids
\item Category of dynamical systems
\end{enumerate}

\subsection{Index of functors}
\emph{The following is a contributed listing of functors:}

\begin{enumerate}
\item Covariant functors
\item Contravariant functors
\item Adjoint functors
\item Preadditive functors
\item Additive functor
\item Representable functors
\item Fundamental groupoid functor
\item Forgetful functors
\item Grothendieck group functor
\item Exact functor
\item Multi-functor
\item Section functors
\item NT2
\item NT3
\end{enumerate}


\subsection{Index of natural transformations}
\emph{The following is a contributed listing of natural transformations:}

\begin{enumerate}
\item Natural equivalence
\item Natural transformations in a 2-category
\item NT3
\item NT1
\item NT2
\item NT3
\end{enumerate}



\subsection{Grothendieck proposals}
\begin{enumerate}
\item \PMlinkname{Esquisse d'un Programme}{AlexSMathematicalHeritageEsquisseDunProgramme}
\item 
\PMlinkexternal{Pursuing Stacks}{http://www.math.jussieu.fr/~leila/grothendieckcircle/stacks.ps}
\item S2
\item S3
\item S4

\end{enumerate}

\subsection{Descent theory}
\begin{enumerate}
\item D1
\item D2
\item D3
\item D4

\end{enumerate}

\subsection{Higher dimensional algebra (HDA)}

\begin{enumerate}
\item Categorical groups 
\item Double groupoids
\item Double algebroids
\item Bi-algebroids
\item $R$-algebroid
\item $2$-category
\item $n$-category
\item Super-category
\item weak n-categories
\item Bi-dimensional Geometry
\item \PMlinkname{Noncommutative geometry}{NoncommutativeGeometry}
\item Higher-Homotopy theories
\item \PMlinkname{Higher-Homotopy Generalized van Kampen Theorem (HGvKT)}{GeneralizedVanKampenTheoremsHigherDimensional}
\item H1
\item H2
\item H3
\item H4

\end{enumerate}



\subsubsection{Axioms of cohomology theory}
\begin{enumerate}

\item A1
\item A2
\item A3
\item A4
\item A5
\item A6
\item A7

\end{enumerate}

\subsubsection{Axioms of homology theory}
\begin{enumerate}

\item A1

\item A2
\item A3
\item A4
\item A5
\item A6

\end{enumerate}

\subsection{Quantum algebraic topology (QAT)} 

\textbf{(a). Quantum algebraic topology} is described as \emph{the mathematical and physical study of  general theories of quantum algebraic structures from the standpoint of algebraic topology, category theory and
their non-Abelian extensions in higher dimensional algebra and supercategories}
\begin{enumerate}
\item Quantum operator algebras (such as: involution, *-algebras, or $*$-algebras, von Neumann algebras,
, JB- and JL- algebras,   $C^*$ - or C*- algebras, 
\item Quantum von Neumann algebra and subfactors; Jone's towers and subfactors
\item Kac-Moody and K-algebras
\item categorical groups 
\item Hopf algebras, quantum Groups and quantum group algebras
\item Quantum groupoids and weak Hopf $C^*$-algebras
\item Groupoid C*-convolution algebras and *-convolution algebroids
\item Quantum spacetimes and quantum fundamental groupoids 
\item Quantum double Algebras
\item Quantum gravity, supersymmetries, supergravity, superalgebras and graded `Lie' algebras
\item Quantum categorical algebra and higher--dimensional, $\L{}-M_n$- Toposes
\item Quantum R-categories, R-supercategories and spontaneous symmetry breaking
\item Non-Abelian Quantum algebraic topology (NA-QAT): closely related to NAAT and HDA.
\end{enumerate}

\subsection{Quantum Geometry}
\begin{enumerate}
\item \PMlinkname{Quantum Geometry overview}{QuantumGeometry2}
\item Quantum non-commutative geometry
\end{enumerate}


\subsection{Non-Abelian Algebraic Topology (NAAT)}

\begin{enumerate}
\item Non-Abelian categories
\item Non-commutative groupoids (including non-Abelian groups)
\item \PMlinkname{Generalized van Kampen theorems}{GeneralizedVanKampenTheoremsHigherDimensional}
\item \PMlinkname{Noncommutative Geometry (NCG)}{NoncommutativeGeometry}
\item Non-commutative `spaces' of functions
\item \PMlinkexternal{non-Abelian Algebraic Topology}{http://planetphysics.org/encyclopedia/NonAbelianAlgebraicTopology5.html}


\end{enumerate}


\subsection{12}


\begin{enumerate}

\item new1

\item new2
\item new3
\item new4

\end{enumerate}


\subsection{13}
\begin{enumerate}

\item new1
\item new2
\item new3
\item new4

\end{enumerate}

\subsection{14}


\subsection{References}

\PMlinkexternal{Bibliography on Category theory, AT and QAT}{http://planetmath.org/?op=getobj&from=objects&id=10746}


\subsubsection{Textbooks and Expositions:}

\begin{enumerate}
\item A \PMlinkexternal{Textbook1}{http://planetmath.org/?op=getobj&from=books&id=172}
\item A \PMlinkexternal{Textbook2}{http://planetmath.org/?op=getobj&from=books&id=156}
\item A \PMlinkexternal{Textbook3}{http://planetmath.org/?op=getobj&from=books&id=159}
\item A \PMlinkexternal{Textbook4}{http://planetmath.org/?op=getobj&from=books&id=160}
\item A \PMlinkexternal{Textbook5}{http://planetmath.org/?op=getobj&from=books&id=153}
\item A \PMlinkexternal{Textbook6}{http://planetmath.org/?op=getobj&from=lec&id=68}
\item A \PMlinkexternal{Textbook7}{http://planetmath.org/?op=getobj&from=books&id=158}
\item A \PMlinkexternal{Textbook8}{http://planetmath.org/?op=getobj&from=lec&id=75}            
\item A \PMlinkexternal{Textbook9}{http://planetmath.org/?op=getobj&from=lec&id=73}
\item A \PMlinkexternal{Textbook10}{http://planetmath.org/?op=getobj&from=books&id=174}
\item A \PMlinkexternal{Textbook11}{http://planetmath.org/?op=getobj&from=books&id=169}
\item A \PMlinkexternal{Textbook12}{http://planetmath.org/?op=getobj&from=books&id=178}
\item A \PMlinkexternal{Textbook13}{http://www.math.cornell.edu/~hatcher/VBKT/VB.pdf}
\item new1


\end{enumerate}

%%%%%
%%%%%
\end{document}
