\documentclass[12pt]{article}
\usepackage{pmmeta}
\pmcanonicalname{FrechetSpace}
\pmcreated{2013-03-22 13:06:10}
\pmmodified{2013-03-22 13:06:10}
\pmowner{rmilson}{146}
\pmmodifier{rmilson}{146}
\pmtitle{Fr\'echet space}
\pmrecord{51}{33528}
\pmprivacy{1}
\pmauthor{rmilson}{146}
\pmtype{Definition}
\pmcomment{trigger rebuild}
\pmclassification{msc}{57N17}
\pmclassification{msc}{54E50}
\pmclassification{msc}{52A07}
\pmrelated{TopologicalVectorSpace}
\pmrelated{HausdorffSpaceNotCompletelyHausdorff}
\pmdefines{F-space}

\endmetadata

\usepackage{amsmath}
\usepackage{amsfonts}
\usepackage{amssymb}
\newcommand{\reals}{\mathbb{R}}
\newcommand{\natnums}{\mathbb{N}}
\newcommand{\cnums}{\mathbb{C}}
\newcommand{\znums}{\mathbb{Z}}
\newcommand{\lp}{\left(} 

\newcommand{\rp}{\right)}
\newcommand{\lb}{\left[}
\newcommand{\rb}{\right]}
\newcommand{\supth}{^{\text{th}}}
\newtheorem{proposition}{Proposition}
\newtheorem{definition}[proposition]{Definition}

\newtheorem{theorem}[proposition]{Theorem}
\begin{document}
We consider two classes of topological vector spaces, one more general
than the other. Following Rudin \cite{rudin_fa} we will define a Fr\'echet space
to be an element of the smaller class, and refer to an instance of the
more general class as an \emph{F-space}. After giving the
definitions, we will explain why one definition is stronger than the
other.


\paragraph{Definition 1.}
An F-space is a complete topological vector space whose topology is
induced by a translation invariant metric. To be more precise, we say
that $U$ is an F-space if there exists a metric function
$$d:U\times U\rightarrow \reals$$
such that $$d(x,y)=d(x+z,y+z),\quad x,y,z\in U;$$
and such that the collection of balls
$$B_\epsilon(x)=\{y\in U: d(x,y)<\epsilon\},\quad x\in U,\;
\epsilon>0$$
is a base for the topology of $U$.

\paragraph{Note 1.} Recall that a topological vector space is a uniform space.
The hypothesis that $U$ is complete is formulated in reference to this
uniform structure. To be more precise, we say that a sequence $a_n\in
U,\; n=1,2,\ldots$ is Cauchy if for every neighborhood $O$ of the
origin there exists an $N\in\natnums$ such that
$a_n-a_m\in O$ for all $n,m>N$. The completeness condition then takes
the usual form of the hypothesis that all Cauchy sequences possess a
limit point.

\paragraph{Note 2.} It is customary to include the hypothesis that $U$ is
Hausdorff in the definition of a topological vector space.
Consequently, a Cauchy sequence in a complete topological space
will have a unique limit.

\paragraph{Note 3.} Since $U$ is assumed to be complete, the pair
$(U,d)$ is a complete metric space. Thus, an equivalent definition of
an F-space is that of a vector space equipped with a complete, translation-invariant (but not necessarily \PMlinkname{homogeneous}{NormedVectorSpace}) metric, such that the operations of scalar
multiplication and vector addition are continuous with respect to this
metric.

\paragraph{Definition 2.}
A Fr\'echet space is a complete topological vector space (either real
or complex) whose topology is induced by a countable family of
semi-norms. To be more precise, there exist semi-norm functions
$$\Vert - \Vert_n : U \rightarrow \reals,\quad n\in\natnums,$$
such that the collection of all balls
$$B_{\epsilon}^{(n)}(x) = \{ y \in U : \Vert x-y\Vert_n < \epsilon\},\quad
x\in U,\; \epsilon>0,\; n\in\natnums,$$
is a base for the topology of $U$.

% A topological vector space is Hausdorff, by definition, and so for
% every non-zero $x\in U$ we must have $\Vert x\Vert_n > 0$ for at least
% one of the seminorms.

\begin{proposition}
Let $U$ be a complete topological vector space. Then, $U$ is a
Fr\'echet space if and only if it is a locally convex F-space.
\end{proposition}

\noindent\emph{Proof.}
First, let us show that a Fr\'echet space is a locally convex F-space,
and then prove the converse. Suppose then that $U$ is Fr\'echet. The
semi-norm balls are convex; this follows directly from the semi-norm
axioms. Therefore $U$ is locally convex. To obtain the desired
distance function we set
\begin{equation}
\label{eq:ddef}
d(x,y) = \sum_{n=0}^\infty 2^{-n} \frac{\Vert x-y \Vert_n}{1+\Vert
x-y\Vert_n},\quad x,y\in U.
\end{equation}

We now show that $d$ satisfies the metric axioms. Let $x,y \in U$
such that $x\neq y$ be given. Since $U$ is Hausdorff, there is at
least one seminorm such
$$\Vert x-y\Vert_n >0.$$
Hence $d(x,y)>0$.

Let $a,b,c>0$ be three real numbers such that
$$a\leq b+c.$$
A straightforward calculation shows
that
\begin{equation}
\label{eq:ineq}
\frac{a}{1+a}\leq \frac{b}{1+b}+\frac{c}{1+c},
\end{equation}
as well. The above trick underlies the definition \eqref{eq:ddef} of
our metric function. By the seminorm axioms we have that
$$\Vert x-z \Vert_n \leq \Vert x-y \Vert_n + \Vert y-z \Vert_n,\quad
x,y,z\in U$$
for all $n$. Combining this with \eqref{eq:ddef} and
\eqref{eq:ineq} yields the triangle inequality for $d$.


Next let us suppose that $U$ is a locally convex F-space, and prove
that it is Fr\'echet. For every $n=1,2,\ldots$ let $U_n$ be an open
convex neighborhood of the origin, contained inside a ball of radius
$1/n$ about the origin. Let $\Vert - \Vert_n$ be the seminorm with
$U_n$ as the unit ball. By definition, the unit balls of these
seminorms give a neighborhood base for the topology of $U$. QED.


\begin{thebibliography}{9}
\bibitem{rudin_fa}
W.Rudin, \emph{Functional Analysis}.
\end{thebibliography}
%%%%%
%%%%%
\end{document}
