\documentclass[12pt]{article}
\usepackage{pmmeta}
\pmcanonicalname{GeneralizedVanKampenSiefertTheoremForDoubleGroupoids}
\pmcreated{2013-03-22 19:23:01}
\pmmodified{2013-03-22 19:23:01}
\pmowner{bci1}{20947}
\pmmodifier{bci1}{20947}
\pmtitle{generalized van Kampen-Siefert theorem for double groupoids}
\pmrecord{10}{42338}
\pmprivacy{1}
\pmauthor{bci1}{20947}
\pmtype{Theorem}
\pmcomment{trigger rebuild}
\pmclassification{msc}{57N20}
\pmclassification{msc}{55Q15}
\pmclassification{msc}{55N35}
\pmsynonym{DGvKST}{GeneralizedVanKampenSiefertTheoremForDoubleGroupoids}
\pmdefines{$\rho$-sequence}
\pmdefines{fundamental group}
\pmdefines{$\pi_1(X}
\pmdefines{x_0)$}
\pmdefines{topological invariant}
\pmdefines{non-commutative invariant}
\pmdefines{fundamental double groupoid}
\pmdefines{morphisms induced by  inclusions}

\endmetadata

% this is the default PlanetMath preamble. as your knowledge
% of TeX increases, you will probably want to edit this, but
\usepackage{amsmath, amssymb, amsfonts, amsthm, amscd, latexsym}
%%\usepackage{xypic}
\usepackage[mathscr]{eucal}
% define commands here
\theoremstyle{plain}
\newtheorem{lemma}{Lemma}[section]
\newtheorem{proposition}{Proposition}[section]
\newtheorem{theorem}{Theorem}[section]
\newtheorem{corollary}{Corollary}[section]
\theoremstyle{definition}
\newtheorem{definition}{Definition}[section]
\newtheorem{example}{Example}[section]
%\theoremstyle{remark}
\newtheorem{remark}{Remark}[section]
\newtheorem*{notation}{Notation}
\newtheorem*{claim}{Claim}
\renewcommand{\thefootnote}{\ensuremath{\fnsymbol{footnote%%@
}}}
\numberwithin{equation}{section}
\newcommand{\Ad}{{\rm Ad}}
\newcommand{\Aut}{{\rm Aut}}
\newcommand{\Cl}{{\rm Cl}}
\newcommand{\Co}{{\rm Co}}
\newcommand{\DES}{{\rm DES}}
\newcommand{\Diff}{{\rm Diff}}
\newcommand{\Dom}{{\rm Dom}}
\newcommand{\Hol}{{\rm Hol}}
\newcommand{\Mon}{{\rm Mon}}
\newcommand{\Hom}{{\rm Hom}}
\newcommand{\Ker}{{\rm Ker}}
\newcommand{\Ind}{{\rm Ind}}
\newcommand{\IM}{{\rm Im}}
\newcommand{\Is}{{\rm Is}}
\newcommand{\ID}{{\rm id}}
\newcommand{\GL}{{\rm GL}}
\newcommand{\Iso}{{\rm Iso}}
\newcommand{\Sem}{{\rm Sem}}
\newcommand{\St}{{\rm St}}
\newcommand{\Sym}{{\rm Sym}}
\newcommand{\SU}{{\rm SU}}
\newcommand{\Tor}{{\rm Tor}}
\newcommand{\U}{{\rm U}}
\newcommand{\A}{\mathcal A}
\newcommand{\Ce}{\mathcal C}
\newcommand{\D}{\mathcal D}
\newcommand{\E}{\mathcal E}
\newcommand{\F}{\mathcal F}
\newcommand{\G}{\mathcal G}
\newcommand{\Q}{\mathcal Q}
\newcommand{\R}{\mathcal R}
\newcommand{\cS}{\mathcal S}
\newcommand{\cU}{\mathcal U}
\newcommand{\W}{\mathcal W}
\newcommand{\bA}{\mathbb{A}}
\newcommand{\bB}{\mathbb{B}}
\newcommand{\bC}{\mathbb{C}}
\newcommand{\bD}{\mathbb{D}}
\newcommand{\bE}{\mathbb{E}}
\newcommand{\bF}{\mathbb{F}}
\newcommand{\bG}{\mathbb{G}}
\newcommand{\bK}{\mathbb{K}}
\newcommand{\bM}{\mathbb{M}}
\newcommand{\bN}{\mathbb{N}}
\newcommand{\bO}{\mathbb{O}}
\newcommand{\bP}{\mathbb{P}}
\newcommand{\bR}{\mathbb{R}}
\newcommand{\bV}{\mathbb{V}}
\newcommand{\bZ}{\mathbb{Z}}
\newcommand{\bfE}{\mathbf{E}}
\newcommand{\bfX}{\mathbf{X}}
\newcommand{\bfY}{\mathbf{Y}}
\newcommand{\bfZ}{\mathbf{Z}}
\renewcommand{\O}{\Omega}
\renewcommand{\o}{\omega}
\newcommand{\vp}{\varphi}
\newcommand{\vep}{\varepsilon}
\newcommand{\diag}{{\rm diag}}
\newcommand{\grp}{{\mathbb G}}
\newcommand{\dgrp}{{\mathbb D}}
\newcommand{\desp}{{\mathbb D^{\rm{es}}}}
\newcommand{\Geod}{{\rm Geod}}
\newcommand{\geod}{{\rm geod}}
\newcommand{\hgr}{{\mathbb H}}
\newcommand{\mgr}{{\mathbb M}}
\newcommand{\ob}{{\rm Ob}}
\newcommand{\obg}{{\rm Ob(\mathbb G)}}
\newcommand{\obgp}{{\rm Ob(\mathbb G')}}
\newcommand{\obh}{{\rm Ob(\mathbb H)}}
\newcommand{\Osmooth}{{\Omega^{\infty}(X,*)}}
\newcommand{\ghomotop}{{\rho_2^{\square}}}
\newcommand{\gcalp}{{\mathbb G(\mathcal P)}}
\newcommand{\rf}{{R_{\mathcal F}}}
\newcommand{\glob}{{\rm glob}}
\newcommand{\loc}{{\rm loc}}
\newcommand{\TOP}{{\rm TOP}}
\newcommand{\wti}{\widetilde}
\newcommand{\what}{\widehat}
\renewcommand{\a}{\alpha}
\newcommand{\be}{\beta}
\newcommand{\ga}{\gamma}
\newcommand{\Ga}{\Gamma}
\newcommand{\de}{\delta}
\newcommand{\del}{\partial}
\newcommand{\ka}{\kappa}
\newcommand{\si}{\sigma}
\newcommand{\ta}{\tau}
\newcommand{\lra}{{\longrightarrow}}
\newcommand{\ra}{{\rightarrow}}
\newcommand{\rat}{{\rightarrowtail}}
\newcommand{\oset}[1]{\overset {#1}{\ra}}
\newcommand{\osetl}[1]{\overset {#1}{\lra}}
\newcommand{\hr}{{\hookrightarrow}}

\begin{document}
\section{The Generalized van Kampen-Siefert Theorem for Double Groupoids (GDvKST)}

This theorem was first published by \PMlinkexternal{Ronald Brown}{http://planetphysics.org/wiki.pl/Ronald_Brown_(mathematician)} and coworkers; please see the cited references.
The following presentation of the GDvKST follows closely Dr. Ronald Brown's presentation.  
\subsection{Double groupoids and connections}
Suppose we are given a cover $\cU$ of $X$. Then the homotopy double groupoids in the following $\rho$-{\em sequence of the cover} are
well-defined:
\begin{equation}
\bigsqcup _{(U,V)  \in \mathcal{U}\, ^2}
 \hdgb(U\cap V)\;
 \overset{a}{\underset{b}{\rightrightarrows}} \bigsqcup _{U \in \cU}\hdgb(U)
 \labto{c} \hdgb(X ) .\label{coeq} 
\end{equation}

The morphisms $a,b$ are determined by the inclusions 
$$a_{UV}:U\cap V\rightarrow U, b_{UV}:U\cap V\rightarrow V$$ for each $(U,V)  \in \mathcal{U}\,^2$ and 
$c$ is determined by the inclusion $c_U:U\rightarrow X$ for each $U \in \cU$.

The next section presents without proof the generalization of the {\em van Kampen-Siefert theorem for double groupoids} that was first proven by Professor Ronald Brown et al; for further details the reader is referred to the original articles listed in the following Bibliography. 

\subsection{GDvKS Theorem for Double Groupoids}

The following is a statement of the Generalized van Kampen Theorem (GvKT) expressed in terms of Double Groupoids with connections as developed and proven in ref. \cite{BHKP}.

\begin{thm}[Generalized van Kampen theorem for double groupoids] 

{\em If the interiors of the sets of $\cU $ cover $X$, then in the above 
$\rho$-sequence of the cover, $c$ is the coequaliser of $a,b$ in the category of double groupoids with connections.} 
\end{thm}

A special case of this result is when $\cU$ has two elements. In this case the coequaliser reduces to a pushout.

{\bf Proof.} The reader is referred to Brown et al.(2004a), \cite{BHKP} for the complete proof of the generalized van Kampen theorem.


Note that this theorem is a generalization of an analogous Van Kampen theorem for the fundamental group \cite {Brown1}. From this theorem, one can compute a particular fundamental group $\pi_1(X,x_0)$ using combinatorial information on the graph of intersections of path components of $U,V,W$, but for this it is useful to develop the algebra of groupoids. Notice two special features of this result:

\noindent (i) The computation of the  {\em topological invariant} one wants to obtain the {\em fundamental group}, is obtained from the computation of a larger structure, and so part of the work is to give {\em methods for computing the smaller structure from the larger one}. This usually involves non-canonical choices, such as that of a maximal tree in a connected graph. The previous work on applying groupoids to groups gives many examples of such methods \cite {brownbook:2}.


\noindent (ii) The fact that the computation can be done is surprising in two ways:
 (a) The fundamental group is computed {\em precisely}, even though the information for 
it uses input in two dimensions, namely $0$ and 1. This is contrary to the experience in
homological algebra and algebraic topology, where the interaction of several dimensions involves 
exact sequences or spectral sequences, which give information only up to extension,  and: 

 (b) the result is a {\em non-commutative invariant}, which is usually even more difficult to 
compute precisely.


The reason for this success seems to be that the fundamental groupoid $\pi_1(X,X_0)$ contains 
information in \emph{dimensions 0 and 1}, and therefore it can adequately reflect the geometry 
of the intersections of the path components of $U,V,W$ and the morphisms induced by the 
inclusions of $W$ in $U$ and $V$. This fact also suggested the question of whether such 
methods could be extended successfully to {\em higher dimensions}.

\begin{thebibliography}{99}

\bibitem{alagl:thesis}
F.~Al-Agl, 1989, {\em Aspects of multiple categories\/}, Ph.D. Thesis, University of Wales, Bangor.

\bibitem{ABS}
F.A. Al-Agl, R. Brown and R. Steiner, Multiple categories:the equivalence of a globular and a cubical approach, 
{\em Advances in Maths.} {\bf 170} (2002) 71-118.

\bibitem{AS}
E. M. Alfsen and F. W. Schultz : \emph{Geometry of Spaces of Operator Algebras}, Birk\"auser, Boston--Basel--Berlin (2003).

\bibitem{Anandan}
J. Anandan : The geometric phase, \emph{Nature} \textbf{360}(1992), 307--313.

\bibitem{antolini:thesis}
R.~Antolini, 1996, {\em Cubical structures and homotopy theory\/},
Ph.D. thesis, Univ. Warwick, Coventry.

\bibitem{ashley}
N.~Ashley, {\em Simplicial $T$-Complexes: a non abelian version of a theorem of Dold-Kan},
Dissertationes Math., 165, (1988), 11 -- 58.

\bibitem{Attal}
R. Attal : Two--dimensional parallel transport: combinatorics and functionality, arXiv:math-ph/0105050

\bibitem{Baez1}
J. Baez. 2004. Quantum quandaries : a category theory perspective, in \emph{`Structural
Foundations of Quantum Gravity}', (ed. S. French et al.) Oxford Univ. Press.

\bibitem{Baez2}
J. Baez. 2002. Categorified Gauge Theory. in Proceedings of the Pacific Northwest Geometry
Seminar Cascade Topology Seminar,Spring Meeting--May 11 and 12, 2002. University of 
Washington, Seattle, WA.

\bibitem{BGGB05}
I.C. Baianu, James Glazebrook, G. Georgescu and Ronald Brown. 2005a.``Generalized 'Topos' 
Represntations of Quantum Space-Time: Linking Quantum N-Valued Logics with Categories and 
Higher Dimensional Algebra.", (\emph{manuscript in preparation})

\bibitem{ICB3} 
I.C. Baianu. 1971a. Organismic Supercategories and Qualitative Dynamics of Systems. The 
Bulletin of Mathematical Biophysics, 33(3):339--354.

\bibitem{ICB4}
I.C. Baianu. 1971b. "Categories, Functors and Quantum Algebraic Computations", Proceed. Fourth 
Intl. Congress LMPS, September 1--4, 1971, Buch.

\bibitem{ICB77}
Baianu, I.C. 1977. ``A Logical Model of Genetic Activities in \Textsl{\L}ukasiewicz Algebras: 
The Non-linear Theory." \emph{Bulletin of Mathematical Biology}, 39:249-258 (1977).

\bibitem{ICB87}
Baianu, I.C. 1987. ``Computer Models and Automata Theory in Biology and Medicine" (A Review). 
In: \emph{"Mathematical Models in Medicine.}",vol.7., M. Witten, Ed., Pergamon Press: New 
York, pp.1513-1577.

 \bibitem{BGGBa}
Baianu, I.C., J. Glazebrook, G. Georgescu and R.Brown. 2005b. ``A Novel Approach to Complex
Systems Biology based on Categories, Higher Dimensional Algebra and A Generalized
\Textsl{\L}ukasiewicz Topos. " , \emph{Manuscript in preparation}, 46 pp.

\bibitem{ICB5} 
I.C. Baianu and D. Scripcariu. 1973.\emph{On Adjoint Dynamical Systems}. \emph{The Bulletin of %%@
Mathematical Biophysics}, \textbf{35}(4):475--486.

\bibitem{Bak}
A. Bak, R. Brown, G. Minian and T. Porter: Global actions, groupoid atlases and related 
topics, http://citeseer.ist.psu.edu/bak00global.html


\bibitem{baues1}
H.~J. Baues, 1989, {\em Algebraic Homotopy\/}, volume~15 of {\em Cambridge  Studies in Advanced Mathematics},
 Cambridge Univ. Press.

\bibitem{babr}
H.~J. Baues and R.~Brown, {\em On relative homotopy groups of the
product filtration, the {J}ames construction, and a formula of {H}opf\/}, J. Pure
  Appl. Algebra, 89, (1993), 49--61, ISSN 0022-4049.

\bibitem{bauesT1}
H.-J. Baues and A.~Tonks, {\em On the twisted cobar construction}, 
Math.  Proc. Cambridge Philos. Soc., 121, (1997), 229--245.

\bibitem{BGV}
N. Berline, E. Getzler and M. Vergne : Heat kernels and Dirac
operators, Grund. der math. Wissenschaften \textbf{298}, Springer
Verlag 1991.

\bibitem{Brown1}
R. Brown : Groupoids and crossed objects in algebraic topology,
\emph{Homology, Homotopy and Appl.} \textbf{1} (1999), 1--78.

\bibitem{Brown04b}
R. Brown. Crossed complexes and homotopy groupoids as non commutative tools for higher 
dimensional local-to-global problems, Proceedings of the Fields Institute Workshop on 
Categorical Structures for Descent and Galois Theory, Hopf Algebras and Semiabelian 
Categories, September 23-28, 2002, Contemp. Math. (2004). 30pp  


\bibitem{BHKP}
R. Brown, K.A. Hardie, K.H. Kamps  and T. Porter, A homotopy double groupoid of a Hausdorff 
space, {\em ``Theory and Applications of Categories"} \textbf{10} (2002) 71-93.

\bibitem{BH} R. Brown  and P.J. Higgins, On the connection between the second relative %%@
homotopy groups of some related spaces,  {\em Proc. London Math. Soc.} \textbf{ (3) 36} %%@
(1978), 193-212.

\bibitem{BH2} R. Brown and P.J. Higgins, On the algebra of cubes, {\em J. Pure Applied %%@
Algebra} \textbf{21 }(1981), 233--260.

\bibitem{colimits} R. Brown  and P.J. Higgins, Colimit theorems for relative homotopy groups', %%@
{\em J. Pure Appl. Algebra} \textbf{22} (1981) 11-41.

\bibitem{BM} 
R. Brown  and G.H. Mosa, Double categories, thin structures and connections,{\it %%@
Theory and Applications of Categories}  {\bf 5} (1999), 163-175.

\bibitem{BS:double}  
R. Brown,  and C.B. Spencer, Double groupoids and crossed modules, {\it 
Cahiers Top. G\'eom.Diff.} {\bf 17} (1976) 343-362.

\bibitem {Brown}
R. Brown.2002.\emph{Categorical Structures for Descent and Galois Theory}. Fields Institute, September 23-28, 2002.

\bibitem{BG1}
R. Brown and J. F. Glazebrook : Connections, local subgroupoids and a holonomy Lie groupoid of a line bundle gerbe, arXiv:math. DG/0210322 \emph{Univ. Iagel. Acta Math.} (2003) (to appear).


\bibitem{BI1}
R. Brown and $\dot{\rm I}$cen : Lie local subgroupoids and their holonomy and monodromy Lie groupoids,
\emph{Top. Appl.} \textbf{115} (2001), 125--138.


\bibitem{BrownBook}
R. Brown: ``{\em Topology: a geometric account of general topology, homotopy types and 
the fundamental groupoid}", Ellis Horwood,Chichester, Prentice Hall, New York, 1988.

\bibitem{BrownBook2}
R. Brown: ``{\em Nonabelian Algebraic Topology, 2011}" ,(\emph{EMS}, 708 pages.

\bibitem{BM}
R. Brown and O. Mucuk : Foliations, locally Lie groupoids and holonomy, \emph{Cah. Top. G\'eom. Diff. Cat.} {\bf 37} (1996), 61--71.


\bibitem{BIM}
R. Brown,  $\dot{\rm I}cen$ and O. Mucuk : Local
subgroupoids II : Examples and properties. \emph{Top. Appl.}
{\bf 127} (2003), 393--408.

\bibitem{brown:gpdsvkt67}
R.~{B}rown, {\em Groupoids and {van Kampen}'s Theorem}, 3,(1967), 385--340.

\bibitem{rb:second80}
R.~Brown, {\em On the second relative homotopy group of an
adjunction space: an exposition of a theorem of {J}. {H}. {C}. {W}hitehead}, J. London Math.
  Soc. (2), 22, (1980), 146--152, ISSN 0024-6107.

\bibitem{rb:coproducts}
R.~Brown, {\em Coproducts of crossed {$P$}-modules: applications
to second homotopy groups and to the homology of groups}, Topology, 23, (1984),337--345.

\bibitem{brown:survey}
R.~Brown, {\em From groups to groupoids: a brief survey}, Bull.London Math.  Soc., 19, (1987), 113--134, ISSN 0024-6093.

\bibitem{brownbook:2}
R.~Brown, 1988, {\em Topology}, Ellis Horwood Series:
Mathematics and its Applications, Ellis Horwood Ltd., Chichester, second edition, a geometric %%@
account of general topology, homotopy types and the fundamental groupoid.

\bibitem{brown:adams}
R.~Brown, 1992, {\em Computing homotopy types using crossed
{$n$}-cubes of  groups}, in {\em Adams Memorial Symposium on Algebraic Topology, 1
  (Manchester, 1990)}, volume 175 of {\em London Math. Soc. Lecture Note
  Ser.\/},  187--210, Cambridge Univ. Press, Cambridge.

\bibitem{br:gilbert}
R.~Brown and N.~D. Gilbert, {\em Algebraic models of {$3$}-types and automorphism structures for crossed modules}, Proc. London Math. Soc. (3),  59, (1989), 51--73, ISSN 0024-6115.

\bibitem{br-gl1}
R.~Brown and J.~F. Glazebrook, {\em Connections, local subgroupoids, and a   holonomy Lie groupoid of a line bundle gerbe}, University of Wales, Bangor,   Math Preprint, 02.22, (2002), 10 pages.

\bibitem{BGPT:Iindag}
R.~Brown, M.~Golasi{\'n}ski, T.~Porter and A.~Tonks, {\em Spaces
of maps into classifying spaces for equivariant crossed complexes}, Indag. Math. (N.S.),
  8, (1997), 157--172, ISSN 0019-3577.

\bibitem{BGPT:IIK-theory}
R.~Brown, M.~Golasi{\'n}ski, T.~Porter and A.~Tonks, {\em Spaces
of maps into classifying spaces for equivariant crossed complexes. {II}. {T}he general %%@
topological group case}, $K$-Theory, 23, (2001), 129--155, ISSN 0920-3036.

\bibitem{bh1978}
R.~Brown and P.~J. Higgins, {\em On the connection between the
second relative homotopy groups of some related spaces\/}, Proc.London Math. Soc., (3) 36, %%@
(1978), 193--212.

\bibitem{bh:CR2}
R.~Brown and P.~J. Higgins, {\em Sur les complexes crois\'es
d'homotopie associ\'es \`a quelques espaces filtr\'es}, C. R. Acad. Sci. Paris S\'er. A-B, %%@
286, (1978), A91--A93, ISSN 0151-0509.

%\bibitem{bh:algcub}
%R.~Brown and P.~J. Higgins, {\em The algebra of cubes\/}, J. Pure
%Appl. Alg., 21, (1981), 233--260.

\bibitem{bh:colimits}
R.~Brown and P.~J. Higgins, {\em Colimit theorems for relative
homotopy groups\/}, J. Pure Appl. Alg, 22, (1981), 11--41.

\bibitem{bh1981a}
R.~Brown and P.~J. Higgins, {\em The equivalence of crossed complexes and  $\omega$-groupoids}, Cahiers Top. G\'{e}om. Diff., 22, (1981), 370 -- 386.

\bibitem{bh:tens}
R.~Brown and P.~J. Higgins, {\em Tensor products and homotopies
for $\omega$-groupoids and crossed complexes}, J. Pure Appl. Alg, 47, (1987),
  1--33.

\bibitem{bh:chn}
R.~Brown and P.~J. Higgins, {\em Crossed complexes and chain complexes with
  operators}, Math. Proc. Camb. Phil. Soc., 107, (1990), 33--57.

\bibitem{bh:class91}
R.~Brown and P.~J. Higgins, {\em The classifying space of a crossed complex},
  Math. Proc. Cambridge Philos. Soc., 110, (1991), 95--120, ISSN 0305-0041.

\bibitem{bh:cubabgp3}
R.~Brown and P.~J. Higgins, 2002, {\em Cubical abelian groups withconnections are equivalent to chain complexes}, Technical Report 02.24.

\bibitem{brownhuebschmann}
R.~Brown and J.~Huebschmann, 1982, {\em Identities among
relations}, in:  R.Brown and T.L.Thickstun, eds., {\em Low Dimensional Topology}, London
  Math. Soc Lecture Notes, Cambridge University Press.

\bibitem{br-icen:2-dhol}
R.~Brown and ~{I}cen, {\em Towards two dimensional holonomy},
Advances in Mathematics,  (to appear).

\bibitem{brownjan:vkt}
R.~Brown and G.~Janelidze.:1997, {\em Van {K}ampen theorems for categories of covering morphisms in lextensive categories}, \emph{J. Pure Appl. Algebra}, \textbf{119}:  255--263, ISSN 0022-4049.

\bibitem{brlo:hur}
R.~Brown and J.-L. Loday, {\em Homotopical excision, and Hurewicz theorems, for $n$-cubes of spaces}, Proc. London Math. Soc., (3)54, (1987), 176 -- 192.

\bibitem{brlo:vkt}
R.~Brown and J.-L. Loday, {\em Van Kampen Theorems for diagrams of spaces\/}, Topology, 26, (1987), 311 -- 337.

\bibitem{brmopw}
R.~Brown, E.~Moore, T.~Porter and C.~Wensley, {\em Crossed
complexes, and free crossed resolutions for amalgamated sums and HNN-extensions of groups},
  Georgian Math. J., 9, (2002), 623--644.

\bibitem{brownmosa}
R.~Brown and G.~H. Mosa, {\em Double categories, {$2$}-categories, thin structures and connections}, Theory Appl. Categ., 5, (1999), No. 7,
  163--175 (electronic), ISSN 1201-561X.

\bibitem{BRazak:LMS99}
R.~Brown and A.~Razak~Salleh, {\em Free crossed resolutions of groups and presentations of modules of identities among relations}, LMS J. Comput.  Math., 2, (1999), 28--61 (electronic), ISSN 1461-1570.

\bibitem{brownspenc:double}
R.~Brown and C.~B. Spencer, {\em Double groupoids and crossed
modules\/}, Cahiers Topologie G\'eom. Diff\'erentielle, 17, (1976), 343--362.

\bibitem{brownspenc:g-gr}
R.~Brown and C.~B. Spencer, {\em $G$-groupoids, crossed modules
and the fundamental groupoid of a topological group}, Proc. Kon. Ned. Akad. v. Wet,
  79, (1976), 296 -- 302.

\bibitem{brownwens:ind}
R.~Brown and C.~D. Wensley, {\em On finite induced crossed modules, and the homotopy {$2$}-type of mapping cones}, Theory Appl. Categ., 1, (1995), No. 3, 54--70 (electronic), ISSN 1201-561X.

\bibitem{brownwens:normal}
R.~Brown and C.~D. Wensley, {\em Computing crossed modules induced by an inclusion of a normal subgroup, with applications to homotopy {$2$}-types},  Theory Appl. Categ., 2, (1996), No. 1, 3--16 (electronic), ISSN 1201-561X.

\bibitem{brownwens:comp}
R.~Brown and C.~D. Wensley, {\em Computations and homotopical
applications of induced crossed modules\/}, J. Symb. Comp.,  (to appear), ISSN 1201-561X.

\bibitem{BungeL}
M.~Bunge and S.~Lack, {\em Van Kampen theorems for toposes}, Advances in Mathematics.

\bibitem{Cech1}
E.~{{\^C}}ech, 1933, {\em H{\"o}herdimensionale
{H}omotopiegruppen\/}, in {\em   Verhandlungen des Internationalen Mathematiker-Kongresses Zurich 1932\/},
volume~2,  203, International Congress of Mathematicians (4th : 1932. Zurich, Switzerland, Walter Saxer, 
Zurich, reprint Kraus, Nendeln, Liechtenstein, 1967.
\end{thebibliography}

%%%%%
%%%%%
\end{document}
