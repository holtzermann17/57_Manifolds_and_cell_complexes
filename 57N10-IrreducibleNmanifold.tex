\documentclass[12pt]{article}
\usepackage{pmmeta}
\pmcanonicalname{IrreducibleNmanifold}
\pmcreated{2013-03-22 16:05:43}
\pmmodified{2013-03-22 16:05:43}
\pmowner{juanman}{12619}
\pmmodifier{juanman}{12619}
\pmtitle{irreducible n-manifold}
\pmrecord{12}{38157}
\pmprivacy{1}
\pmauthor{juanman}{12619}
\pmtype{Definition}
\pmcomment{trigger rebuild}
\pmclassification{msc}{57N10}

\endmetadata

% this is the default PlanetMath preamble.  as your knowledge
% of TeX increases, you will probably want to edit this, but
% it should be fine as is for beginners.

% almost certainly you want these
\usepackage{amssymb}
\usepackage{amsmath}
\usepackage{amsfonts}

% used for TeXing text within eps files
%\usepackage{psfrag}
% need this for including graphics (\includegraphics)
%\usepackage{graphicx}
% for neatly defining theorems and propositions
%\usepackage{amsthm}
% making logically defined graphics
%%%\usepackage{xypic}

% there are many more packages, add them here as you need them

% define commands here

\begin{document}
An \PMlinkname{$n$-manifold}{TopologicalManifold} $M$ is called
\emph{irreducible} if for each embedding of a standard $(n-1)$-sphere $S^{n-1}$ in
$M$, there is an embedding of a \PMlinkname{standard $n$-ball}{StandardNBall} $D^n$ in $M$ such that the
image of the boundary $\partial D^n$ coincides with the image of
$S^{n-1}$.

In case of dimension three it can be proved that each irreducible
3-manifold is also a \PMlinkname{prime}{Prime3Manifold} 3-manifold.

\PMlinkescapeword{irreducible}
%%%%%
%%%%%
\end{document}
